
\subsubsection*{General information}


\begin{itemize}
\itemsep0em
\item The recommended tool in this course is R (with the IDE R-Studio). You can download R \href{https://cran.r-project.org/}{\textbf{here}} and R-Studio \href{https://www.rstudio.com/products/rstudio/download/}{\textbf{here}}. You are allowed to use Python and Jupyter Notebooks, although the assignments may use data available only through the R package, a problem you would need to solve yourself.

\item  Report all results in a single, \texttt{*.pdf}-file. \emph{Other formats, such as Word, Rmd, Jupyter Notebook, or similar, will automatically be failed.} Although, you are allowed to first knit your document to Word and then print the assignment as a PDF from Word if you find it difficult to get TeX to work.

\item The report should be submitted to the \href{https://studium.uu.se/}{\textbf{Studium}}.

\item To pass the assignments, \emph{all questions should be answered}, although minor errors are ok.

\item A report that do not contain the general information (see template) will be automatically rejected.

\item When working with R, we recommend writing the reports using R markdown and the provided \href{https://raw.githubusercontent.com/MansMeg/IntroML/master/templates/assignment_template.rmd}{\textbf{R markdown template}}. The template includes the formatting instructions and how to include code and figures.

\item If you have a problem with creating a PDF file directly from R markdown, start by creating an HTML file, and then just print the HTML to a PDF.

\item Instead of R markdown, you can use other software to make the pdf report, but the same instructions for formatting should be used. These instructions are also available in \href{https://raw.githubusercontent.com/MansMeg/IntroML/master/templates/assignment_template.pdf}{\textbf{the PDF produced from the R markdown template}}.

\item The course has its own R package \texttt{uuml} with data and functionality to simplify coding. To install the package just run the following:
\begin{enumerate}
\item \texttt{install.packages("remotes")}
\item \texttt{remotes::install\_github("MansMeg/IntroML", \\ subdir = "rpackage")}
\end{enumerate}

%\item Many of the exercises can be checked automatically using the R package \\ \texttt{markmyassignment}. Information on how to install and use the package can be found \href{https://cran.r-project.org/web/packages/markmyassignment/vignettes/markmyassignment.html}{\textbf{here}}. There is no need to include \texttt{markmyassignment} results in the report.

\item We collect common questions regarding installation, and technical problems in a course Frequently Asked Questions (FAQ). This can be found \href{https://github.com/MansMeg/IntroML/blob/master/FAQ.md}{\textbf{here}}.

\item Deadline for all assignments is \textbf{Sunday at 23.59}. See the course page for dates.

\item If you have any suggestions or improvements to the course material, please post in the course chat feedback channel, create an issue, or submit a pull request to the public repository!

\end{itemize}

