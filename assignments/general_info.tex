
\subsubsection*{General information}

\begin{itemize}
\itemsep0em
\item The recommended tool in this course is R (with the IDE R-Studio). You can download R \href{https://cran.r-project.org/}{\textbf{here}} and R-Studio \href{https://www.rstudio.com/products/rstudio/download/}{\textbf{here}}. You can use Python and Jupyter Notebooks, although the assignments may use data available only through the R package, a problem you would need to solve yourself.

\item  Report all results in a single, \texttt{*.pdf}-file. \emph{Other formats, such as Word, Rmd, Jupyter Notebook, or similar, will automatically be failed.} Although, you are allowed first to knit your document to Word or HTML and then print the assignment as a PDF from Word or HTML if you find it difficult to get TeX to work.

\item The report should be written in English.

\item If a question is unclear in the assignment. Write down how you interpret the question and formulate your answer.

\item You should submit the report to \href{https://studium.uu.se/}{\textbf{Studium}}. Deadlines for all assignments are \textbf{Sunday 23.59}. See \textbf{Studium} for dates. Assignments will be graded within 10 working days from the assignment deadline.

\item To pass the assignments, \emph{you should answer all questions not marked with *}, and get at least 75\% correct.

\item To get VG on the assignment, \emph{all questions should be answered, including questions marked with a *}, although minor errors are ok. VG will only be awarded on the first deadline of the assignment.

\item A report that does not contain the general information (see the \href{https://raw.githubusercontent.com/MansMeg/IntroML/master/templates/assignment_template.pdf}{\textbf{template}}), will be automatically rejected.

\item When working with R, we recommend writing the reports using R markdown and the provided \href{https://raw.githubusercontent.com/MansMeg/IntroML/master/templates/assignment_template.rmd}{\textbf{R markdown template}}. The template includes the formatting instructions and how to include code and figures.

\item Instead of R markdown, you can use other software to make the pdf report, but you should use the same instructions for formatting. These instructions are also available in \href{https://raw.githubusercontent.com/MansMeg/IntroML/master/templates/assignment_template.pdf}{\textbf{the PDF produced from the R markdown template}}.

\item The course has its own R package \texttt{uuml} with data and functionality to simplify coding. To install the package just run the following:
\begin{enumerate}
\item \texttt{install.packages("remotes")}
\item \texttt{remotes::install\_github("MansMeg/IntroML", \\ subdir = "rpackage")}
\end{enumerate}

%\item Many of the exercises can be checked automatically using the R package \\ \texttt{markmyassignment}. Information on how to install and use the package can be found \href{https://cran.r-project.org/web/packages/markmyassignment/vignettes/markmyassignment.html}{\textbf{here}}. There is no need to include \texttt{markmyassignment} results in the report.

\item We collect common questions regarding installation and technical problems in a course Frequently Asked Questions (FAQ). This can be found \href{https://github.com/MansMeg/IntroML/blob/master/FAQ.md}{\textbf{here}}.

\item You are not allowed to show your assignments (text or code) to anyone. Only discuss the assignments with your fellow students. The student that show their assignment to anyone else could also be considered to cheat. Similarly, on zoom labs, only screen share when you are in a separate zoom room with teaching assistants.

\item The computer labs are for asking all types of questions. Do not hesitate to ask! The purpose of the computer labs are to improve your learning. We will hence focus on more computer labs and less on assignment feedback. \emph{Warning!} There might be bugs in the assignments! Hence, it is important to ask questions early on so you dont waste time of unintentional bugs.

\item If you have any suggestions or improvements to the course material, please post in the course chat feedback channel, create an issue \href{https://github.com/MansMeg/IntroML/issues}{\textbf{here}}, or submit a pull request to the public repository.

\end{itemize}

