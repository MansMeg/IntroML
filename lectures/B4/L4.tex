%\documentclass[10pt,handout]{beamer}
\documentclass[10pt]{beamer}
\usepackage{babel} % Anpassa efter svenska. Ger svensk logga.
\usepackage[utf8]{inputenc} % Anpassa efter linux
\usepackage{graphicx}
\usepackage{../common/beamerthemeUppsala}
%\usecolortheme{UU} % Anpassa efter UU:s frger och logga
%\hypersetup{pdfpagemode=FullScreen} % Adobe Reader ska ppna fullskrm
\setbeamertemplate{itemize items}[circle]

% \usepackage{beamerthemesplit}
\usepackage{amsmath}
\usepackage{amssymb}
% \usepackage{graphics}
% \usepackage{graphicx}
% \usepackage{epsfig}
% \usepackage[latin1]{inputenc}
 \usepackage{color}
% \usepackage{fancybox}
% \usepackage{psfrag}
% \usepackage[english]{babel}
 \setbeamertemplate{footline}{\hfill\insertframenumber/\inserttotalframenumber}


%library(tinytex)
%tlmgr_install('csquotes')
%\usepackage{csquotes}

% Read in commands
% Course settings
\newcommand{\currentsemester}{Autumn 2023}

% New commands
\newcommand{\bfm}[1]   {\mbox{\boldmath{${#1}$}}}
\newcommand{\Prob}   {\mbox{\textnormal{P}}}
\newcommand{\uured}[1]{\textcolor{uured}{#1}}

% Eqds
\def\eqd{\,{\buildrel d \over =}\,}

% Math operators
\DeclareMathOperator{\E}{\mathbb{E}}


%%%%%%%%%%%%%%%%%%%%%%%%%%%%%%%%%%%%%%%%%%%%%%%%%%%%%%%%%%%%%%%%%%

\setlength{\parskip}{3mm}
\title[]{{\color{black}Machine learning -- Block 4}}
\author[]{M{\aa}ns Magnusson\\Department of Statistics, Uppsala University}
\date{\currentsemester}


\begin{document}

\frame{\titlepage
% \thispagestyle{empty}
}

%%%%%%%%%%%%%%%%%%%%%%%%%%%%%%%%%%%%%%%%%%%%%%%%%%%%%%%%%%%%%%%%%%


\begin{frame}{This week's lecture}
\begin{itemize}
\item Previous assignments
\item The Mini-Project and Master Thesis Projects
\item Convolutional Neural Networks
\item Transfer Learning
\end{itemize}
\end{frame}



%%%%%%%%%%%%%%%%%%%%%%%%%%%%%%%%%%%%%%%%%%%%%%%%%%%%%%%%%%%%%%%%%%

\section{Practicalities}

\begin{frame}{Assignment 3: Evaluation}

%\begin{itemize}
%\item A little bit to easy VG part.
%\item How to think when building networks? Art and Science. Radom search and %try to get as low training error as possible.
%\item Keras lecture? Describe parameters?
%\end{itemize}

\end{frame}

\begin{frame}{On this weeks assignment}
\begin{itemize}
\item It takes long time to run the models this week. Start early!
\end{itemize}
\end{frame}


\section{Introduction}
\frame{\sectionpage}

\frame{\frametitle{Convolutional Neural Networks}

\begin{itemize}
\item \uured{Acknowledgements}: Anders Eklund, Link\"{o}ping University.\pause
\item \uured{Convolutional} Neural Networks are behind great progress in the 2010s.
\item Revolutionized \uured{Computer Vision}.
\pause
\item Also called: ConvNets, Convolutional nets, Convolutional networks
\end{itemize}

\begin{figure}[h]
\caption{ImageNet performance (Roessler, 2019)}
\centering
\includegraphics[width=0.8\textwidth]{fig/imageNet.png}
\end{figure}
\pause

}


\frame{\frametitle{Convolutional Neural Networks}

\begin{itemize}
\item Special architecture that works well for data with a \uured{grid structure}
\begin{enumerate}
\item 1D-grids: Time series\pause
\item 2D-grids: Gray-scale Images (pixels)\pause
\item 3D-grids: Color Images (pixels and chanels)\pause
\item 4D-grids: Color Video (pixels, chanels, frames)
\end{enumerate}
\end{itemize}

}

\frame{\frametitle{Computer Vision}

\begin{itemize}
\item Problems
\begin{itemize}
\item Image Classification\pause
\item Image Segmentation\pause
\item Object Detection
\item ...\pause
\end{itemize}
\item {\color{uured} Focus}: 2D and 3D data\pause
\item Very Large Datasets:
\begin{itemize}
\item ImageNet: 14M Images, 20k classes, 1M bounding boxes
\end{itemize}
\pause
\item Many different pre-trained models (e.g. VGG16)
\end{itemize}
}

\frame{\frametitle{Example: Object Detection}
\begin{figure}[h]
\centering
\includegraphics[width=0.8\textwidth]{fig/object_detection.png}
\caption{Object detection (see \href{https://www.youtube.com/watch?v=VOC3huqHrss}{https://www.youtube.com/watch?v=VOC3huqHrss}) }
\end{figure}
}

\frame{\frametitle{Example: Pneumonia detection}
\begin{figure}[h]
\centering
\includegraphics[width=0.8\textwidth]{fig/chexnet.png}
\caption{Rajpurkar et al. (2017). Chexnet: Radiologist-level pneumonia detection on chest x-rays with deep learning. arXiv preprint arXiv:1711.05225.}
\end{figure}
}

\frame{\frametitle{Example: Fracture detection}
\begin{figure}[h]
\centering
\includegraphics[width=0.8\textwidth]{fig/fractures.png}
\caption{Olczak et al, (2017) Artificial intelligence
for analyzing orthopedic trauma radiographs, Acta Orthopaedica, 88:6, 581-586}
\end{figure}
}


\frame{\frametitle{What is an Image?}

\begin{itemize}
\item 2-dimensional object
\item Each pixel has:
\begin{enumerate}
\item a coordinate
\item a value (light intensity)
\end{enumerate}\pause
\item {\color{uured} Grayscale}: single channel
\item {\color{uured} Color}: three channel (RGB)\pause
\item Spatial and hiearchical correlation structures
\end{itemize}
}

\frame{\frametitle{MNIST example}

\begin{figure}[h]
\centering
\includegraphics[width=0.7\textwidth]{fig/MNIST_example.png}
\caption{Example from the MNIST dataset (28 by 28 pixels)}
\end{figure}

}

\frame{\frametitle{How to train models for images?}

\begin{itemize}
\item We want to learn {\color{uured} representations} of parts of images
\begin{figure}[h]
\centering
\includegraphics[width=0.7\textwidth]{fig/DLR_Fig_5_2_cat.png}
\caption{The representations of a cat (Chollet and Allair, 2018, Fig 5.2)}
\end{figure}
\pause
\item CNN uses \uured{Convolutional Layers} to learn \uured{parameter efficient} representations
\end{itemize}

}


\frame{\frametitle{Learning Representations for Images (again)}

\begin{figure}[h]
\centering
\includegraphics[width=0.8\textwidth]{fig/DL_fig_1_2_representations.png}
\caption{Learning representations for images (Goodfellow et al, 2017, Fig. 1.2)}
\end{figure}

}


\section{Convolution}
\frame{\sectionpage}


\frame{\frametitle{Convolution}

\begin{itemize}
\item Different definitions are common, one example:
\[
y(t) = \int x(\tau) k(t-\tau)d\tau = (x*k)(t)
\]
\item Inutuition: "Weighting together two functions"
\item In a convolutional layer:
\begin{enumerate}
\item $x(t)$: Input
\item $k(t)$: Kernel, filter, "feature"
\item $y(t)$: Output, feature map
\end{enumerate}
\vspace{5mm}
\centering
\href{https://phiresky.github.io/convolution-demo/}{DEMO}

\end{itemize}
}


\frame{\frametitle{Discrete Convolution}

\begin{itemize}
\item If $t$ is discrete (as in a grid):
\[
y(t) = (x*k)(t) = \sum_{\tau=-{\infty}}^{\infty} x(\tau) k(t-\tau)
\]
\pause
\item In the case of images we have 2 discrete dimensions
\[
Y(i,j) = (X * K) \sum_{m} \sum_{n} X(m, n) K(i-m, j-n)
\]
\pause
\item Sometimes the \uured{cross-correlation} is called convolution:
\[
Y(i,j) = (X * K) \sum_{m} \sum_{n} X(m, n) K(i+m, j+n)
\]
\begin{enumerate}
\item $X(i,j)$: Input (2D)
\item $K(i,j)$: Kernel, filter, "feature" (2D)
\item $Y(i,j)$: Output, feature map (2D)
\end{enumerate}
\end{itemize}
}



\frame{\frametitle{Convolution of Images: 2D}
\begin{figure}[h]
\centering
\includegraphics[width=0.8\textwidth]{fig/DL_fig_9_1_conv.png}
\caption{Convolution for an Image (Goodfellow et al, 2017, Fig. 9.1)}
\end{figure}
}

\frame{\frametitle{Convolution of images: Example}
\begin{figure}[h]
\centering
\includegraphics[width=0.8\textwidth]{fig/2d_conv_eklund.png}
\caption{Convolution example.}
\end{figure}
}

\frame{\frametitle{Convolution of images: Examples}
\begin{figure}[h]
\centering
\includegraphics[width=0.8\textwidth]{fig/DLR_fig_5_3_conv_example.png}
\caption{Convolution for an Image (Chollet and Allaire, 2018, Fig. 5.3)}
\end{figure}

}


\frame{\frametitle{Convolution of images: Example}
\[
X=
  \begin{bmatrix}
    0 & 0 & 0 & 1 & 1 \\
    0 & 0 & 0 & 1 & 1 \\
    0 & 0 & 0 & 1 & 1 \\
    0 & 0 & 1 & 1 & 1
  \end{bmatrix}\,,
  K =
  \begin{bmatrix}
    -1 & 1 \\
    -1 & 1
  \end{bmatrix}
\]
\pause
\[
Y=
  \begin{bmatrix}
    0 & 0 & 2 & 0 \\
    0 & 0 & 2 & 0 \\
    0 & 1 & 2 & 0
  \end{bmatrix}\,,
\]
}

\frame{\frametitle{Edge Detection}
\begin{figure}[h]
\centering
\includegraphics[width=0.95\textwidth]{fig/Fig_10_4_BB.png}
\caption{Edge Detections (Bishop and Bishop, 2024, Fig. 10.4)}
\end{figure}
}



\section{Convolutional Neural Networks}
\frame{\sectionpage}

\frame{\frametitle{Convolutional Neural Networks}
\begin{figure}[h]
\centering
\includegraphics[width=0.8\textwidth]{fig/DL_fig_9_7_conv_layer.png}
\caption{Convolution layer (Goodfellow et al, 2018, Fig. 9.7)}
\end{figure}
}

\frame{\frametitle{Convolutional Neural Networks}

\begin{itemize}
\item Most convolutional neural networks have:
\begin{enumerate}
\item Many convolutional layers
\item More kernels higher up in the network
\item A classification head (usually a feed-forward neural network)
\end{enumerate}
\pause
\item Benefits:
\begin{enumerate}
\item Few(er) parameters (filters)
\item Captures {\color{uured} local structures}
\item Efficient computations\pause
\end{enumerate}
\item How to choose filters?
\begin{enumerate}
\item Before: {\color{uured} manually handcrafted}\pause
\item Now: {\color{uured} learn the filters}
\end{enumerate}

\end{itemize}

}

\subsection{The Convolution Layer}

\frame{\frametitle{Convolution layer}

\begin{itemize}
\item {\color{uured} Input}: Data or Feature Maps
\pause
\item {\color{uured} Parameters}:
\begin{itemize}
\item $N$ filters/kernels of size $m\times m$
\item $N$ bias terms (one per filter)
\end{itemize}
\pause
\item {\color{uured} Activation functions}: Applied element wise on feature maps
\pause
\item {\color{uured} Output}: Feature Maps
\pause
\item In Keras:\\
\texttt{layer\_conv\_2d(filters = 32, kernel\_size = c(3,3), activation = "relu", input\_shape = c(32,32,3))}
\end{itemize}

}

\frame{\frametitle{Padding}

\begin{itemize}
\item Handling {\color{uured} edges}
\item \emph{Padding}: add 0 around the image
\item Nessecary to {\color{uured} keep size} of feature maps
\end{itemize}

}


\frame{\frametitle{Padding}

\begin{figure}[h]
\centering
\includegraphics[width=0.8\textwidth]{fig/DLR_fig_5_5_valid.png}
\includegraphics[width=0.8\textwidth]{fig/DLR_fig_5_6_padded.png}
\caption{Padding and valid edge handling (Chollet and Allair (2018), Fig. 5.5, 5.6)}
\end{figure}
}


\frame{\frametitle{Stride}

\begin{itemize}
\item {\color{uured} Skip} every $n$th pixel
\item Reduces the computations
\end{itemize}

\begin{figure}[h]
\centering
\includegraphics[width=0.8\textwidth]{fig/DLR_fig_5_7_2g2_stride.png}
\caption{Strides (Chollet and Allair (2018), Fig. 5.5, 5.6)}
\end{figure}
}


\frame{\frametitle{Why Convolution Layers?}

\begin{itemize}
\item Captures local spatial structure
\item Reduces the number of parameters (parameter sharing)
\begin{enumerate}
\item The number and size of filters
\item We use the same filters everywhere\pause
\end{enumerate}
\item Example: a 1 megapixel image ($1000 \times 1000$ pixels)
\begin{enumerate}
\item Dense network with 100 nodes: {\color{uured} 100M} parameters
\item CNN network with 100 $3\times 3$ filters: {\color{uured} 1000} parameters \\(900 from filters, 100 bias terms)
\end{enumerate}
\end{itemize}
}

\frame{\frametitle{Convolution Neural Nets}
\begin{figure}[h]
\centering
\includegraphics[width=0.8\textwidth]{fig/DL_fig_9_7_conv_layer.png}
\caption{Convolution layer (Goodfellow et al, 2018, Fig. 9.7)}
\end{figure}
}


\frame{\frametitle{Detector stage}

\begin{itemize}
\item Remember, in feed-forward networks: $\mathbf{h} = \sigma(\mathbf{X W}  + b)$
\item In CNN:
\begin{enumerate}
\item $\mathbf{W}$ is the filter\pause
\item $\mathbf{X}$ is the input feature map\pause
\item $\mathbf{X W}$ is the convolutional feature map\pause
\item $b$ is a bias (one per filter) \pause
\item $\sigma$ is the activation function (usually a ReLU)
\end{enumerate}
\end{itemize}

}

\subsection{The Pooling Layer}
\frame{\frametitle{Pooling layer}

\begin{itemize}
\item We take a function $f$ that return one value per pooling kernel\pause
\item Most commonly $f=\max$
\item Commonly a $2\times 2$ pooling kernel with stride 2
\item {\color{uured} Why?} Reduce the size of feature map, but keep the activation\pause
\item In Keras:\\
\texttt{layer\_max\_pooling\_2d(pool\_size = c(2, 2))}(
\end{itemize}

}

\frame{\frametitle{Max Pooling}

\begin{figure}[h]
\centering
\includegraphics[width=0.8\textwidth]{fig/MaxpoolSample2.png}
\caption{Strides (Computer Science Wikipedia)}
\end{figure}
}

\frame{\frametitle{Using pooling to learn invariances}

\begin{itemize}
\item pooling over spatial positions: invariant to translation
\item pooling over different filters: invariant to transformations
\end{itemize}

\begin{figure}[h]
\centering
\includegraphics[width=0.7\textwidth]{fig/invariances.png}
\caption{Learning invariances (Goodfellow et al., 2017, Fig. 9.9)}
\end{figure}
}

\subsection{Regularization}

\frame{\frametitle{Data Augmentation}
\begin{figure}[h]
\centering
\includegraphics[width=0.6\textwidth]{fig/DLR_fig_5_10_data_augmentation.png}
\caption{Data Augmentation (Chollet and Allair, 2018, Fig 5.10)}
\end{figure}
\pause
\begin{itemize}
\item Can be done directly in Keras (data generator)
\end{itemize}
}


\subsection{Examples}

\frame{\frametitle{Popular CNN architectures}

\begin{itemize}
\item AlexNet (2012), 5 convolutional layers
\item VGG16 (2014), 16 convolutional layers
\item ResNet (2015), 152 convolutional layers
\end{itemize}
}

\frame{\frametitle{VGG16}

\begin{figure}[h]
\centering
\includegraphics[width=0.9\textwidth]{fig/CNNVGG.png}
%\caption{Strides (Computer Science Wiki: "Max-pooling / Pooling")}
\end{figure}

}

\section{Transfer learning}
\frame{\sectionpage}

\frame{\frametitle{Transfer learning}

\begin{itemize}
\item "{\color{uured} Transfer knowledge} between problems"
\item \uured{Learning representations} in $P_1$ will aid \uured{generalization} in $P_2$\pause
\item A Bayesian perspective: A \uured{strong prior}
\end{itemize}

}



\frame{\frametitle{Learning Representations for Images}

\begin{figure}[h]
\centering
\includegraphics[width=0.8\textwidth]{fig/DL_fig_1_2_representations.png}
\caption{Learning representations can be crucial (Goodfellow et al, 2017, Fig. 1.2)}
\end{figure}

}

\frame{\frametitle{Transfer learning}

\begin{itemize}
\item \uured{In practice}: Transfer/reuse {\color{uured} learned weights} or rather \uured{some weights}\pause
\item Use (large) \uured{pre-trained} models for smaller problems
\item A reason for the success of CNN\pause
\item Two types of transfer learning in Neural Networks:
\begin{itemize}
\item Feature extraction (use pre-trained networks for features)
\item Fine Tuning (adapt pre-trained features)
\end{itemize}
\end{itemize}

}

\frame{\frametitle{Feature Extraction}

\begin{figure}[h]
\centering
\includegraphics[width=0.8\textwidth]{fig/DLR_fig_5_12_feature_extract.png}
\caption{Using convnets as base for feature extraction (Chollet and Allair, 2018, Fig 5.12)}
\end{figure}

}

\frame{\frametitle{Fine-Tuning}

\begin{figure}[h]
\centering
\includegraphics[width=0.45\textwidth]{fig/DLR_fig_5_15_a_fine_tune.png}\includegraphics[width=0.45\textwidth]{fig/DLR_fig_5_15_b_fine_tune.png}
\caption{Finetuning a convolutional base (Chollet and Allair, 2018, Fig 5.15)}
\end{figure}

}


\frame{\frametitle{Transfer learning}

\begin{itemize}

\item \uured{Catastophic forgetting} \pause
\item \uured{Domain adaptation}: Same problem but at different input dataset\\ (e.g. language models for legal/medical/political data) \pause
\item \uured{Concept drift}: Similar problem \pause
\item Previously, popular with unsupervised pre-training.
\end{itemize}

}


\section{Practical Methodology}
\frame{\sectionpage}

\frame{\frametitle{Practical Methodology}

\begin{enumerate}
\item Determine your goals\pause
\item Setup your baseline \\(establish a working end-to-end pipeline as soon as possible)\pause
\item Diagnose your networks performance\pause
\item Make incremental improvements
\end{enumerate}

\begin{itemize}
\item \uured{General idea}: Increase data and model capacity until goal is reached\pause
\item \uured{End goal}: Good enough performance on test set
\end{itemize}

}


\frame{\frametitle{1. Determine your goals}

\begin{itemize}
\item Why are you building a model?\pause
\item Set up the metric based on the \uured{overall goal} of the system! \\This may need multiple metrics.\pause %
\item What is good enough? Remember the \uured{Bayes error}!\pause
\item What performance can you expect?\pause
\item Some errors are worse than others, e.g. spam filters.\pause
\item Handling of uncertain predictions:
\item \uured{Coverage}: How large proportions can the system predict?\pause
\item Manual curation can be faster and easier.
\end{itemize}
}

\frame{\frametitle{2. Setup your baseline}

\begin{itemize}
\item Start with...
\begin{itemize}
\item the most simple possible model (logistic regression)\pause
\item previous approaches/baselines\pause
\item a simple neural network that is common in the domain/defaults\\(CNN for images, Adam as optimizer)
\end{itemize}
\end{itemize}
}


\begin{frame}{Remember: Statistical Learning Theory}

\begin{figure}[h]
\caption{Test, training, and model complexity (Goodfellow et al, 2017, Figure 5.3)}
\centering
\includegraphics[width=0.8\textwidth]{fig/Dl_5_3.png}
\end{figure}

\end{frame}

\frame{\frametitle{3. Diagnose and improve}

\begin{itemize}
\item High training loss: Training data not fully used\\
Neural network generally performs best when training error is low
\pause
\item High test loss: Low data quality, i.e. large Bayes error?\pause
\item Low training error and high test error: Common situation\pause
\begin{itemize}
\item You can always improve by gather more data, or
\item Regularize to optimal capacity\pause
\end{itemize}
\item \uured{How to know the improvements of additional data?}
\end{itemize}
}

\frame{\frametitle{Learning curve}

\begin{figure}[h]
\centering
\includegraphics[width=0.8\textwidth]{fig/DL_5_4a.png}
\caption{Learning curve to assess the need for more data (Goodfellow et al., 2017, Fig 5.4)}
\end{figure}

}

\frame{\frametitle{3. Diagnosing and improving your model: Regularization}

\begin{itemize}
\item Best performance: Larger model that is regularized well.
\item Warning: Avoid \uured{the algorithm rabbit hole}\pause
\item Adapt hyperparameters to get \uured{optimal capacity}\pause
\item Hyperparameter Search Goal: \\adjust the model capacity to match the complexity of the task.\pause
\item Marginal hyperparameter has a U-shaped error function (ideally)\pause
\item Neural Networks Steps:
\begin{enumerate}
\item Get good training error. E.g. by tune learning rate and increasing capacity\pause
\item Tune hyperparameters (regularization):\\
Requires monitoring both training and test error
\end{enumerate}
\end{itemize}
}

\frame{\frametitle{Hyper-parameter optimization}

\begin{itemize}
\item Neural networks has many hyperparameters
\item \uured{What is a hyperparameter in a neural network?}\pause
\item Grid search:
\begin{itemize}
\item Setup a grid of potential values
\item For 1-4 hyperparameter: grid search can work well
\item Usually iterative/repeated grid search is best: start with three values, and work iteratively
\item Grows exponentially with the number of parameters
\end{itemize}
\pause
\item Random search
\begin{itemize}
\item Specify a marginal distribution for each hyperparameter (additional work)
\item Common: uniform on log scale
\item Can also be done iteratively
\item Can be exponentially more efficient
\item Generally reduce the error faster in setting with many hyper parameters
\end{itemize}
\end{itemize}
}

\frame{\frametitle{Diagnosis and Debugging}

\begin{itemize}
\item Visualize the model predictions\pause
\item Analyze the worst errors: \uured{why?}\pause
\item Use train and test error as a diagnostic: \uured{Can you overfit the data?}\pause
\item Use test suites: Can you get known results on a toy data. Both training error and derivatives\pause
\item Monitor the gradients\pause
\item Monitor activation function statistics. Are some never activated?
\end{itemize}
}

\end{document}
