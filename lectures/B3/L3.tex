%\documentclass[10pt,handout]{beamer}
\documentclass[10pt]{beamer}
\usepackage[english]{babel} % Anpassa efter svenska. Ger svensk logga.
\usepackage[utf8]{inputenc} % Anpassa efter linux
\usepackage{graphicx}
\usepackage{../common/beamerthemeUppsala}
%\usecolortheme{UU} % Anpassa efter UU:s frger och logga
%\hypersetup{pdfpagemode=FullScreen} % Adobe Reader ska ppna fullskrm
\setbeamertemplate{itemize items}[circle]

% \usepackage{beamerthemesplit}
\usepackage{amsmath}
\usepackage{amssymb}
% \usepackage{graphics}
% \usepackage{graphicx}
% \usepackage{epsfig}
% \usepackage[latin1]{inputenc}
 \usepackage{color}
% \usepackage{fancybox}
% \usepackage{psfrag}
% \usepackage[english]{babel}
 \setbeamertemplate{footline}{\hfill\insertframenumber/\inserttotalframenumber}


% Read in commands
% Course settings
\newcommand{\currentsemester}{Autumn 2023}

% New commands
\newcommand{\bfm}[1]   {\mbox{\boldmath{${#1}$}}}
\newcommand{\Prob}   {\mbox{\textnormal{P}}}
\newcommand{\uured}[1]{\textcolor{uured}{#1}}

% Eqds
\def\eqd{\,{\buildrel d \over =}\,}

% Math operators
\DeclareMathOperator{\E}{\mathbb{E}}


%%%%%%%%%%%%%%%%%%%%%%%%%%%%%%%%%%%%%%%%%%%%%%%%%%%%%%%%%%%%%%%%%%

\setlength{\parskip}{3mm}
\title[]{{\color{black}Machine learning -- Block 3}}
\author[]{M{\aa}ns Magnusson\\Department of Statistics, Uppsala University}
\date{\currentsemester}


\begin{document}

\frame{\titlepage
% \thispagestyle{empty}
}



%%%%%%%%%%%%%%%%%%%%%%%%%%%%%%%%%%%%%%%%%%%%%%%%%%%%%%%%%%%%%%%%%%


\section{Practicalities}

\begin{frame}{Evaluation assignment 2}
\begin{itemize}
\item t
\end{itemize}
\end{frame}


\begin{frame}{This week's lecture}
\begin{itemize}
\item Feed-Forward Neural Networks
\item Regularization of Neural Networks
\item Neural Network Optimization
\end{itemize}
\end{frame}


\section{Introduction}
\frame{\sectionpage}

%%%%%%%%%%%%%%%%%%%%%%%%%%%%%%%%%%%%%%%%%%%%%%%%%%%%%%%%%%%%%%%%%%


\frame{\frametitle{The Hype: Computer Vision}

\begin{figure}[h]
\caption{ImageNet performance (Roessler, 2019)}
\centering
\includegraphics[width=0.8\textwidth]{fig/imageNet.png}
\end{figure}
}


\frame{\frametitle{The Hype: Speech Recognition}

\begin{figure}[h]
\centering
\includegraphics[width=0.8\textwidth]{fig/SP.jpeg}
\caption{Speech recognition performance (source: \url{https://eff.org/ai/metrics})}
\end{figure}
}

\frame{\frametitle{The Hype: Natural Language Processing}

\begin{figure}[h]
\centering
\includegraphics[width=0.8\textwidth]{fig/GLUE.png}
\caption{General Language Understanding (source: \url{https://www.programmersought.com/article/4251948498/})}
\end{figure}

Work is very much ongoing:

\url{https://gluebenchmark.com/leaderboard}


}

\frame{\frametitle{The Hype}

\begin{itemize}
\item Although - Neural Networks is not a silver bullet\pause
\item Remember the {\color{uured} Bayes error}\pause
\item Some times a linear regression (or Random Forest) is enough
\end{itemize}

}


\section{Feed-Forward Neural Networks}
\frame{\sectionpage}

%\subsection{Learning} % And backprop

%\subsection{Hidden Units}

%\subsection{Architecture design}


\frame{\frametitle{The Feed-Forward Network}

\begin{figure}[h]
\centering
\includegraphics[width=0.8\textwidth]{fig/simple_NN.png}
\caption{A simple feed-forward network (Goodfellow et al, 2017)}
\end{figure}

{\color{uured} Important concepts}:

Layers, neurons, input, output, weights, bias, architecture

}

\frame{\frametitle{Different Architectures for Different Purposes}

\begin{itemize}
\item Different networks for different purposes
\begin{itemize}
\item {\color{uured} Convolutional Neural Networks}: Computer Vision\pause
\item {\color{uured} Recurrent Neural Networks}: Speech Audio (?)\pause
\item {\color{uured} Transformers/Attention}: Textual data
\end{itemize}
\item The Neural Network Zoo: \url{https://www.asimovinstitute.org/neural-network-zoo/}
\end{itemize}

}


\frame{\frametitle{Areas of Use: All fields}

\begin{itemize}
\item Supervised learning
\item Unsupervised learning
\item Reinforcement learning
\end{itemize}

}


\frame{\frametitle{Why and when neural nets?}

\begin{itemize}
\item Learning feature representations
\item Needs a lot of data to learn complex representations
\item Good for sensor data (high-dimensional)\pause
\item When should we not use neural networks?
\end{itemize}

}

\frame{\frametitle{Learning Representations}

\begin{figure}[h]
\centering
\includegraphics[width=0.8\textwidth]{fig/DL_fig_1_2_representations.png}
\caption{Learning representations can be crucial (Goodfellow et al, 2017, Fig. 1.2)}
\end{figure}

}


\subsection{Feed-Forward Neural Networks}
\frame{\frametitle{The Feed-Forward Network}

\begin{figure}[h]
\centering
\includegraphics[width=0.8\textwidth]{fig/simple_NN.png}
\caption{A simple feed-forward network (Goodfellow et al, 2017, Fig. 6.2)}
\end{figure}

In mathematical notation:
\[
y_i = \mathbf{w}^T g(\mathbf{W}^T\mathbf{x}_i + \mathbf{b}_1) + \mathbf{b}_2
\]

}

\frame{\frametitle{The Feed-Forward Network}

\[
y_i = \mathbf{w}^T g(\mathbf{W}^T\mathbf{x}_i + \mathbf{b}_1) + \mathbf{b}_2
\]

\begin{equation*}
W =
\begin{pmatrix}
1 & 1 \\
1 & 1 \\
\end{pmatrix}
\,,w =
\begin{pmatrix}
1 \\
-2 \\
\end{pmatrix}\,,
b_1 =
\begin{pmatrix}
1 \\
-1 \\
\end{pmatrix}\,,
b_2 =
\begin{pmatrix}
0 \\
\end{pmatrix}
\end{equation*}

\begin{equation*}
g(z) = ReLU(z) = \max(0,z)\,,
x_i =
\begin{pmatrix}
0 \\
0 \\
\end{pmatrix}\,,
\end{equation*}

\[
y_i = \begin{pmatrix}
1 \\
-2 \\
\end{pmatrix}^T g(\begin{pmatrix}
0 \\
0 \\
\end{pmatrix} +
\begin{pmatrix}
1 \\
-1 \\
\end{pmatrix}
) + \begin{pmatrix}
0 \\
\end{pmatrix}
\]
\[
y_i = \begin{pmatrix}
1 \\
-2 \\
\end{pmatrix}^T \begin{pmatrix}
1 \\
0 \\
\end{pmatrix} + \begin{pmatrix}
0 \\
\end{pmatrix} = 1
\]

}



\frame{\frametitle{The Feed-Forward Network}

A feed-forward network for one observation ($x_i$).

\begin{align*}
\underbrace{\mathbf{h}_1}_{1 \times k_1} &= g_1(\underbrace{\mathbf{x}^{T}}_{1 \times p}\underbrace{\mathbf{W}_1}_{p \times k_1} + \underbrace{\mathbf{b}_1}_{1 \times k_1}) \\
& \vdots \\
\underbrace{\mathbf{h}_l}_{1 \times k_l}  &=  g_l(\underbrace{\mathbf{h}_{l-1}^{T}}_{1 \times k_{l-1}}\underbrace{\mathbf{W}_l}_{k_{l-1} \times k_l} + \underbrace{\mathbf{b}_l}_{1 \times k_l}) \\
& \vdots \\
\underbrace{\hat{\mathbf{y}}}_{1 \times m} &= g_{L}(\underbrace{\mathbf{h}^{T}_{{L-1}}}_{1 \times k_{l-1}}\underbrace{\mathbf{W}_{L}}_{k_{l-1} \times m} + \underbrace{\mathbf{b}_{L}}_{1 \times m})
\end{align*}

\[
\hat{y} = f_L(f_{L-1}(...f_1(x)...))
\]

% TODO Fix mathbf

}


\frame{\frametitle{Activation functions ($g_l$)}

\begin{itemize}
\item Sometimes use notation $\sigma$ as in $\sigma(W h + b)$\pause
\item Historically $g(z)$ has been the sigmoid or or hyperbolic tangent (tanh)
\[
g_\text{sigmoid}(z)={\frac {e^{z}}{e^{z}+1}}={\frac {1}{1+e^{-z}}}
\]
\[
g_\text{tanh}(z)={\frac {\sinh z}{\cosh z}}={\frac {e^{2z}-1}{e^{2z}+1}}
\]
\pause
\item Now, usually variants of Rectified linear unit (ReLU)
\[
g_\text{ReLU}(z)=\max({0,z})
\]
\begin{itemize}
\item Easier to estimate with SGD
\item Easier for deep models
\end{itemize}
\item Last activation is the output function $g_L$, usually a softmax (if classification)
\[
f(z_i)={\frac {e^{z_i}}{\sum _{j=1}^{J}e^{z_{j}}}}
\]
\end{itemize}

}



\frame{\frametitle{Activation functions ($g_l$)}

\begin{figure}[h]
\centering
\includegraphics[width=0.6\textwidth]{fig/ReLu.pdf}
\caption{Rectified Linear Unit (Goodfellow et al, 2017, Fig. 6.3)}
\end{figure}

}


\frame{\frametitle{Universal Approximation Theorem}

"A feed-forward neural network with a linear output layer and at least one hidden layer with any 'squashing' activation function can approximate any Borel measurable function from one finite-dimensional space to another with any desired non-zero amount of error, provided that the network is given enough hidden units." (Goodfellow et al. 2017, p. 198)

\begin{itemize}
\item Also holds for ReLU
\item No garantuee we can learn the network
\item No garantuee that it will generalize
\item No indication of how large the network need to be
\end{itemize}

}

\subsection{Hyper-parameters}

\frame{\frametitle{Hyper-parameters in feed-forward networks}

\begin{itemize}
\item The number of layers
\item The number of neurons
\item Activation functions
\item The type of layers (CNN, MaxPooling, Multi-head attention)
\end{itemize}

}

\frame{\frametitle{How to choose parameters}

\begin{itemize}
\item Trial and error on validation sets
\item Art rather than science
\item Specialized approaches (Bayesian Optimization)\pause
\item Grid search (combinatorical explosion)
\begin{itemize}
\item Really bad with many parameters with less effects
\item If we have 5 irrelevant parameters we try 3 values for: 125 training per relevant run
\item Instead use...\pause
\end{itemize}
\item Random search
\end{itemize}

}

\frame{\frametitle{Grid search vs. Random Search}

\begin{figure}[h]
\centering
\includegraphics[width=0.6\textwidth]{fig/random_search.png}
\caption{Grid search and random search (Goodfellow et al, 2017, Fig. 11.2)}
\end{figure}

}




\section{Regularization}
\frame{\sectionpage}

%\subsection{Norm penalty}
% 7.1

%\subsection{Dataset augmentation}
% 7.4

%\subsection{Multi-Task Learning}
% 7.7

%\subsection{Early Stopping}
% 7.8

%\subsection{Drop-out}
% 7.12

\frame{\frametitle{Regularization of Neural Networks}

\begin{itemize}
\item Reduce traing error but improve test/validation error\pause
\item Neural Networks are extremely flexible / high model capacity\pause
\item Regularization is crucial for good generalizability of NN\pause
\end{itemize}

}

\frame{\frametitle{Weight decay / Norm penalty}

\begin{itemize}
\item Let
\[
\tilde{J}(W,b) = J(W,b) + \alpha \Omega(W)\,,
\]
where $J(W,b)$ is the cost function and $\alpha \Omega(W)$ is the penalty for the weight matrices.
\item $\alpha$ is the strength of the penalty.
\end{itemize}

}

\frame{\frametitle{Weight decay / Norm penalty}

\begin{itemize}
\item Let
\[
\Omega_1 (W) = \sum_i \sum_j |w|_{i,j}\,,
\]
and
\[
\Omega_2 (W) = \sum_i \sum_j w^2_{i,j}\,,
\]
be the $L_1$ and $L_2$ regularization respectively.
\item We can then get the cost function
\[
\tilde{J}(W,b) = J(W,b) + \sum_l \alpha_l \Omega_2(W_l)\,,
\]
\end{itemize}

}


\frame{\frametitle{Weight decay / Norm penalty}

\begin{itemize}
\item Lets define the cost function as
\begin{align*}
\tilde{J}(w) &= J(w) + \alpha \Omega_2(w)\\
             &= J(w) + \alpha w^T w
\end{align*}
\item Then the gradient update becomes
\begin{align*}
\nabla_w \tilde{J}(w) &= \nabla_w J(w) + 2\alpha w
\end{align*}
\item To update our weights with gradient descent
\begin{align*}
w \leftarrow & w - \epsilon( \nabla_w J(w) + 2\alpha w)\\
w \leftarrow & (1 - 2\alpha \epsilon)w - \epsilon \nabla_w J(w)\\
\end{align*}
\end{itemize}

}


\frame{\frametitle{Weight decay / Norm penalty}

\begin{figure}[h]
\centering
\includegraphics[width=0.6\textwidth]{fig/L2.png}
\caption{$L_2$ regularization (Goodfellow et al, 2017, Fig. 7.1)}
\end{figure}

}

\frame{\frametitle{Early Stopping}

\begin{itemize}
\item Stop optimization early based on validation error
\item Rerun to that number of epochs (hyperparameter)
\item Can be shown to be quivalent (under strict assumptions) to $L_2$ regularization
\end{itemize}

\begin{figure}[h]
\centering
\includegraphics[width=0.8\textwidth]{fig/early_stopping.png}
\caption{Early Stopping (Goodfellow et al, 2017, Fig. 7.3)}
\end{figure}
}

\frame{\frametitle{Early Stopping}

\begin{figure}[h]
\centering
\includegraphics[width=0.8\textwidth]{fig/early_stopping_L2.png}
\caption{Early Stopping (Goodfellow et al, 2017, Fig. 7.4)}
\end{figure}
}

\frame{\frametitle{Dropout}

\begin{itemize}
\item In each iteration:
\begin{itemize}
\item Sample an indicator $I_i$ for each node $i$
\item Set the value $h_i$ to 0 with probability $p$
\end{itemize}
\item The dropout probability is typically 0.8 for input nodes and 0.5 for hidden nodes
\item Forces the network to
\begin{itemize}
\item not rely on individual nodes
\item spread out the weights over more nodes
\end{itemize}
\item Can be seen as an ensamble method
\end{itemize}
}


\frame{\frametitle{Dropout}


\begin{figure}[h]
\centering
\includegraphics[width=1\textwidth]{fig/dropout.png}
\caption{Dropout (Srivastava et al, 2014)}
\end{figure}

}


\frame{\frametitle{Other regularization techniques}


\begin{itemize}
\item In CNN: Dataset augmentation
\item Get more data...
\end{itemize}

}

\section{Optimization of Neural Networks}
\frame{\sectionpage}

\frame{\frametitle{Neural Network Learning}

\begin{itemize}
\item Usually, a lot of data and many parameters ($\theta = (W, b)$)\pause
\item We usually minimize our training cost function
\[
J(\theta) = \sum_i^N L(\text{NN}(x_i),y_i) + \Omega(\theta)\,,
\]
where $L$ is the observation level loss, $\text{NN}()$ is our neural network and $\Omega$ is the regularization term. \pause
\item \uured{Learning Target}: Find $\hat{\theta}$ that minimize the \uured{generalization} error \pause
\end{itemize}

}


\frame{\frametitle{Optimization of Neural Networks}

\begin{itemize}
\item Stochastic Gradient Descent methods
\[
\theta_t = \theta_{t-1} - \eta_t \hat{\nabla} J(\theta_{t-1})
\]
\item In practice, better optimizers are used:
\begin{itemize}
\item Adam
\item RMSprop
\end{itemize}\pause
\item To compute gradients ($\hat{\nabla} J(\theta_{t-1})$):
\begin{itemize}
\item \uured{Backpropagation} algorithm (chain-rule for derivatives)
\end{itemize}
\end{itemize}

% Double check that backprop is introduced before/elaborate more in backprop

}


%\subsection{Ill-Conditioning}
% 8.2.1

\subsection{Local minima}
% 8.2.2
% Neural Networks are (usually) not a convex problem
% This mean that we can have local minima
% this is a problem if we get stuck in a local minima with high cost
% This is probably not the case
% Many practitioners believe that this is a problem
% Diagnose: Plot the gradient norm (gTg) over epochs

%\subsection{Plateaus and Saddle Points}
% 8.2.3
% https://en.wikipedia.org/wiki/Saddle_point
% Another problem is saddle points gradient is what?
% Many random functions have saddle points in high dimensions
% A saddle point is a gradient of 0 Hessian is indefinite i.e. both positive and negative eigen values
% A minimum has a positive definite Hessian.
% To understand the intuition behind this behavior, observe that the Hessian matrix at a local minimum has only positive eigenvalues. The Hessian matrix at a saddle point has a mixture of positive and negative eigenvalues. Imagine that the sign of each eigenvalue is generated by flipping a coin.

% An amazing property of many random functions is that the eigenvalues of the Hessian become more likely to be positive as we reach regions of lower cost. In our coin tossing analogy, this means we are more likely to have our coin come up heads n times if we are at a critical point with low cost. This means that local minima are much more likely to have low cost than high cost. Critical points with high cost are far more likely to be saddle points.

% The proliferation of saddle points in high dimensional spaces presumably explains why second-order methods have not succeeded in replacing gradient descent for neural network training.

% On the other hand, gradient descent empirically seems to be able to escape saddle points in many cases.

%\subsection{Cliffs, exploding and vanishing gradients}
% 8.2.4-8.2.5
% Copy image 8.2.3

%\subsection{Inexact and difficult gradients}
% 8.2.6
% TODO




\frame{\frametitle{Problem}

\begin{itemize}
\item Difficult problem
\item Many local minima (weight space symmetry)
\item Platueas and sadel points
\begin{itemize}
\item Gradient is small - but not a minimum or maximum
\item Sadel points increases with the number of dimensions (?)
\item Large areas with small change in cost function
\end{itemize}
\end{itemize}

}


\subsection{Parameter initialization}
% 8.4

\frame{\frametitle{Initial values}
\begin{itemize}
\item We need to have \uured{starting values} for gradient descent\pause
\item Initialization can be seen as a hyperparameter\pause
\item Bad initial values might
\begin{itemize}
\item Bad convergence (local optimum)
\item Numerical problems
\end{itemize}\pause
\item We want to \uured{break symmetry} between layers
\begin{itemize}
\item Otherwise the same units will be updated in the same way (deterministically)
\end{itemize}
\pause
\item Good practice
\begin{itemize}
\item Initialize values randomly close to zero (uniform or normal)
\end{itemize}
\end{itemize}

% TODO: readup more in breaking symmetry

}


%\frame{\frametitle{Batch normalization}
%
%\begin{itemize}
%\item A way to simplify the optimization
%\end{itemize}
% TODO: Fix this part - add information here on batch normalization
%}

\section{Neural Networks in Practice}
\frame{\sectionpage}

\frame{\frametitle{TensorFlow}
\begin{itemize}
\item Framework for large-scale machine learning and Neural Networks
\item Developed by Google
\item Can be used both from R and Python
\item Used in both research and production
\item What Tesorflow does:
\begin{itemize}
\item Computing gradients (autodiff) for Neural Networks
\item Enable use of graphical processing units (GPU) and Tensor processing Units (TPU)
\item Enable training using common optimizers (such as Adam, RMSprop)
\end{itemize}
\item Tesorflow Probability is a probabilistic programming framework using TF
\end{itemize}
\centering
\includegraphics[width=0.2\textwidth]{fig/TF_logo.png}

}


\frame{\frametitle{(Py)Torch}

\begin{itemize}
\item Similar to TensorFlow
\item Developed by Meta AI
\item Can be used both from R and Python
\item Used in both research and production
\item \uured{pyro} is a probabilistic programming framework using torch
\end{itemize}

\centering
\includegraphics[width=0.5\textwidth]{fig/pytorch_logo.png}

}

\frame{\frametitle{Keras}

\begin{itemize}
\item Syntax for 'building' Neural Networks
\item Available both in R and Python
\item TensorFlow or Torch as backend
\end{itemize}

% TODO: Add an image of KERAS here and how to read a model

\centering
\includegraphics[width=0.2\textwidth]{fig/Keras_logo.png}

}

\end{document}
